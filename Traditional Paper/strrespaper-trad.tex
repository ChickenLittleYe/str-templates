\documentclass{strrespaper-trad}
\usepackage[utf8]{inputenc}
\usepackage{csquotes}
\usepackage[english]{babel}

% For better tables
\usepackage{booktabs}

% For images
\usepackage{graphicx}

% For subfigures
\usepackage{subfig}

% For landscape pages
\usepackage{pdflscape}

% For drawing
\usepackage{tikz}
\usetikzlibrary{shapes.multipart}

% For Project Plan
\usepackage{pgfgantt}
\usepackage{tabularx}

% For code listings (type "texdoc listings" [without the quotes] for more information)
\usepackage{listings}

% For species names (type "texdoc biocon" [without the quotes] for more information)
\usepackage{biocon}
\newplant{At}{genus=Arabidopsis,epithet=thaliana,author=Heynh.,oldauthor=L.}
\newbact{ecoli}{genus=Escherichia, epithet=coli}

% Feel free to remove once your remove all the dummy text
\usepackage{lipsum} 

% Required for the citations
\usepackage[style=apa,sortcites=true,sorting=nyt,backend=biber]{biblatex}
\addbibresource{../sample/bibliographies/str.bib} % Add your BibLaTeX files here

 
\title{RESEARCH PROJECT TITLE} % Put your proper title here
\dayofmonth{13} % Depends on prof
\date{May 2019} % Also depends on prof

\tptitle{
	(UPPERCASE except for \textit{Scientific names}) \\
	RESEARCH PROJECT TITLE
} % Format your title to make it an inverted pyramid

% Names in alphabetical order (First M.I. Family Name)
\addAuthor{Researcher D. One} 
\addAuthor{A\~na B. Two}
\addAuthor{Scholar N. Bayan}

\adviser{Adviser C. Adviser} % Name of adviser
\level{3} % STR Year Level


% Setup for PDF Metadata
\makeatletter
\hypersetup{
	pdftitle={\@title},
	pdfauthor={Researcher One, A\~na Two, Scholar Bayan},
	pdfkeywords={stemr, research, paper, template}
}
\makeatother


\begin{document}
    \maketitle

    \makeapprovalsheet

    \makeacknowledgement{
        We would like to thank everyone included in the study for their time and effort. Without them, this research would not be successful.
    }

    \makeabstract{
        \lipsum[1-2]
    }

    \contents
    \listoflistings

    \mainmatter

    \section{INTRODUCTION}
	    \subsection{Background of the Study}
	        \lipsum[1]

	    \subsection{Objectives of the Study}
	        \subsubsection{General Objective}
	            \begin{itemize}
	                \item This is a general objective that's supposed to describe exactly what your research will do. (Filler text is nice)
	            \end{itemize}
	        \subsubsection{Specific Objectives}
	            \begin{enumerate}
	                \item Specific
	                \item Another specific
	                \item Some specific
	            \end{enumerate}

	    \subsection{Significance of the Study}
	        \lipsum[2]

	        Random paragraph for \plant{At}. \plant[e]{At} is a plant. \plant{At} is composed of plant stuff.

	        Random paragraph for \bact{ecoli}. \bact{ecoli} is a form of bacteria. \bact{ecoli} is made of bacteria stuff. (Random listing shown in Listing~\ref{lst:game.py})

	    \subsection{Scope and Limitations of the Study}
	        \lipsum[4]

    \definitionofterms{}
        \termdef{Banana}{
            I am putting this as senseless text. This is not a definition. Absolutely nothing here makes sense, and that's completely the point.
        }
        \termdef{Definition}{
            something you place here in the Definition of Terms to clarify your paper
        }

    \section{REVIEW OF RELATED LITERATURE}
	    \paragraph{Vestibulum}
	        \lipsum[5-9]

	        Random line for \bact[f]{ecoli}.

    \section{MATERIALS AND METHODS}
	    \subsection{Research Design}
	        \lipsum[10]

	    \subsection{Locale of the Study}
	        \textcite{letcher_wind_2017} states the following: \lipsum[11].

	    \subsection{Materials and Research Instruments}
	        \lipsum[12]

	    \subsection{Procedures}
	        \lipsum[13]

	    \subsection{Treatment of Data}
	        \subsubsection{Statement of Hypotheses} \vspace{-2em}
	            \begin{align*}
	                H_0: & \mu = 50 \\
	                H_a: & \mu > 50
	            \end{align*}
	        \subsubsection{Analysis of Data}
	            \lipsum[4]

    \section{RESULTS AND DISCUSSION}
	    \paragraph{Results}
	        \lipsum[13]
	        \begin{figure}[ht]
	            \centering
	            \includegraphics[width=1\linewidth]{../sample/graphics/anonymous.png}
	            \caption[Anonymous]{Anonymous}
	            \label{fig:anonymous}
	        \end{figure}

	        \lipsum[15]

	        \begin{table}[ht]
	            \centering
	            \caption[Sample Table]{Sample Table}
	            \begin{tabular}{cccc}
	                \toprule
	                foo & bar & yeet & skeet \\
	                \midrule
	                1   & 12  & 122  & 11    \\
	                3   & 4   & 11   & 33    \\
	                \bottomrule
	            \end{tabular}
	            \label{tab:sampletable}
	        \end{table}

	        \lipsum[14]

	        \begin{eqnarray}
	            E = mc^2 \label{eq:meeq} \\
	            Yeet \label{eq:yeet}
	        \end{eqnarray}

	        The solution to Equation~\ref{eq:meeq} is Equation~\ref{eq:yeet} and is shown in Figure~\ref{fig:anonymous} and Table~\ref{tab:sampletable} \autocite{al-shemmeri_wind_2010}

    \section{CONCLUSION AND RECOMMENDATIONS}
	    \lipsum[2-4]

    \literaturecited{}

    \appendix

    \section{PROJECT PLAN}
	    \begin{table}[htbp]
	        \centering
	        \caption{Task Lists and Duration}
	        \label{tab:task_lists_duration}
	        \begin{tabularx}{\linewidth}{cXcc}
	            \toprule
	            Task & Task Description                                                      & Preceding Tasks & Duration (in days) \\
	            \midrule
	            A    & Development of Warzone Game Platform and Implementation of Algorithms & ---             & 30                 \\
	            B    & Testing, Refinement and Optimization of Implemented Programs          & A               & 31                 \\
	            C    & Data Collection                                                       & B               & 60                 \\
	            D    & Data Analysis                                                         & C               & 60                 \\
	            \bottomrule
	        \end{tabularx}
	    \end{table}

	    \begin{figure}[htbp]
	        \centering
	        \newcommand{\netchart}[8]{
	            \node
	            [rectangle split,
	                rectangle split parts = 3,
	                draw,
	                minimum width = 3cm,
	                font = \small,
	                rectangle split part align = {center}
	            ] (#8)
	            {
	                \centering
	                \begin{tabularx}{2.75cm}{@{} >{\centering\arraybackslash}X|>{\centering\arraybackslash}X|>{\centering\arraybackslash}X @{}}
	                    #1 & #2 & #3 \\
	                \end{tabularx}
	                \nodepart{two}
	                #4
	                \nodepart{three}
	                \begin{tabularx}{2.75cm}{@{} >{\centering\arraybackslash}X|>{\centering\arraybackslash}X|>{\centering\arraybackslash}X @{}}
	                    #5 & #6 & #7 \\
	                \end{tabularx}
	            };
	        }
	        \newcommand{\nchconnect}[2]{\draw[->] (#1.two east) to [out=0, in=180] (#2.two west);}
	        \tikzset{
	            block/.style={
	                    rectangle,
	                    draw,
	                    minimum height=4em,
	                    minimum width=4em,
	                    text depth=1em,
	                    outer sep=0pt,
	                    font=\small
	                }
	        }
	        % !!! Disclaimer: shift distances are exaggerated to demonstrate functionality for more complex networks e.g. multiple threads. If your flow is linear, simply omit the yshift components
	        \begin{tikzpicture}
	            \netchart{0}{30}{30} {Task A} {0}{30}{30} {tA}
	            \begin{scope}[xshift=4cm]
	                \begin{scope}[yshift=2cm]
	                    \netchart{30}{31}{61} {Task B} {30}{31}{61} {tB}
	                    \begin{scope}[xshift=4cm, yshift=-4cm]
	                        \netchart{61}{60}{121} {Task C} {61}{60}{121} {tC}
	                    \end{scope}
	                \end{scope}
	                \begin{scope}[xshift=8cm]
	                    \netchart{121}{60}{181} {Task D} {121}{60}{181} {tD}
	                \end{scope}
	            \end{scope}
	            \nchconnect{tA}{tB}
	            \nchconnect{tB}{tC}
	            \nchconnect{tC}{tD}
	        \end{tikzpicture}
	        \caption{Network Chart}
	        \label{fig:network_chart}
	    \end{figure}


	    \begin{landscape}
	        \newpage
	        \newcommand{\tsmpap}[8]{\makecell{#1} & #2 & \makecell{#3} & \makecell{#4} & \makecell{#5} & \makecell{#6} & \makecell{#7} & \makecell{#8} \\}
	        \newenvironment{TSMPAP}{\tabularx{\linewidth}{cXcccccc}}{\endtabularx}
	        \begin{table}[ht]
	            \centering
	            \caption{Task Schedule Management and Personnel Assignment Plan}
	            \label{tab:task_schedule_personnel_assignment}
	            \begin{tabular}{cccccccc}
	                \toprule
	                Task & Task Description           & Personnel                  & Duration (in days) & EST                        & LST                        & ECT                        & LCT                        \\
	                \midrule
	                A    & \begin{tabular}{c} Development of Warzone Game Platform and \\ Implementation of Algorithms \end{tabular} & \begin{tabular}{c}All\\Personnel\end{tabular} & 30                 & \begin{tabular}{c} NOV\\01\\2019 \end{tabular} & \begin{tabular}{c} NOV\\30\\2019 \end{tabular} & \begin{tabular}{c} NOV\\01\\2019 \end{tabular} & \begin{tabular}{c} NOV\\30\\2019 \end{tabular} \\
	                B    & \begin{tabular}{c} Testing, Refinement and Optimization \\ of Implemented Programs \end{tabular} & \begin{tabular}{c}All\\Personnel\end{tabular} & 31                 & \begin{tabular}{c} DEC\\01\\2019 \end{tabular} & \begin{tabular}{c} DEC\\31\\2019 \end{tabular} & \begin{tabular}{c} DEC\\01\\2019 \end{tabular} & \begin{tabular}{c} DEC\\31\\2019 \end{tabular} \\
	                C    & Data Collection            & \begin{tabular}{c}All\\Personnel\end{tabular} & 60                 & \begin{tabular}{c} JAN\\01\\2019 \end{tabular} & \begin{tabular}{c} FEB\\29\\2019 \end{tabular} & \begin{tabular}{c} JAN\\01\\2019 \end{tabular} & \begin{tabular}{c} FEB\\29\\2019 \end{tabular} \\
	                D    & Data Analysis              & \begin{tabular}{c}All\\Personnel\end{tabular} & 61                 & \begin{tabular}{c} MAR\\01\\2019 \end{tabular} & \begin{tabular}{c} APR\\31\\2019 \end{tabular} & \begin{tabular}{c} MAR\\01\\2019 \end{tabular} & \begin{tabular}{c} APR\\31\\2019 \end{tabular} \\
	                \bottomrule
	            \end{tabular}
	        \end{table} \newpage

	        \newcommand{\mesp}[6]{#1 & \makecell{#2} & \makecell{#3} & \makecell{#4} & \makecell{#5} & \makecell{#6} \\}
	        \newenvironment{MESP}{\tabularx{\linewidth}{Xccccc}}{\endtabularx}
	        \begin{table}[htbp]
	            \centering
	            \caption{Material and Equipment Sourcing Plan}
	            \label{tab:material_equipment_sourcing}
	            \begin{tabular}{cccccc}
	                \toprule
	                Protocol                     & Date/s Needed    & Unit & Materials Needed         & Potential Source & Remarks \\
	                \midrule
	                \begin{tabular}{c} Development of Warzone Game Platform \\ and Implementation of Algorithm \end{tabular}   & NOV-01 to 30     & 1    & Laptop with Python       & From Home        & On Hand \\
	                \begin{tabular}{c} Testing, Refinement and Optimization \\ of Implemented Programs \end{tabular}   & DEC-01 to 31     & 1    & Laptop with Python       & From Home        & On Hand \\
	                Data Collection and Analysis & JAN-01 to APR-31 & 1    & Laptop with Python and R & From Home        & On Hand \\
	                \bottomrule
	            \end{tabular}
	        \end{table}

	        \begin{table}[htbp]
	            \centering
	            \caption{Risk Management Plan}
	            \label{tab:risk_management}
	            \begin{tabularx}{\textwidth}{cc}
	                \toprule
	                Risk                                  & Safety Measure/Protocol                     \\
	                \midrule
	                Development of Carpal Tunnel Syndrome & Frequent 5-minute breaks to relieve muscles \\
	                Electrocution                         & Proper usage of electronic devices          \\
	                Loss of data                          & Data upload into the cloud                  \\
	                Proprietary software trial expiry     & Use of free and open-source software        \\
	                \bottomrule
	            \end{tabularx}
	        \end{table}
	    \end{landscape}

    \section{RAW DATA}
	    \begin{table}[htbp]
	        \centering
	        \begin{tabular}{cc}
	            \toprule
	            Random & Bits \\
	            \midrule
	            of     & data \\
	            \bottomrule
	        \end{tabular}
	        \caption{Random Table}
	        \label{tab:random}
	    \end{table}
	    \lstset{language=Python, breaklines=true, numbers=left, stringstyle=\ttfamily\small, basicstyle=\singlespacing}

	    \lstinputlisting[caption={game.py}, label={lst:game.py}]{../sample/code/game.py}

    \section{STATISTICAL TABLES}
	    \begin{table}[htbp]
	        \centering
	        \begin{tabular}{lccc}
	            \toprule
	            Aspect           & Human & Fish & Bacterium \\
	            \midrule
	            opposable thumbs & 1     & 0    & 0         \\
	            wings            & 0     & 0    & 0         \\
	            gills            & 0     & 1    & 0         \\
	            eyes             & 1     & 1    & 0         \\
	            requires oxygen  & 1     & 1    & 1         \\
	            \bottomrule
	        \end{tabular}
	        \caption{Example of Organisms Sorted Through a Classifier System}
	        \label{tab:ex_classifier}
	    \end{table}

    \section{DOCUMENTATION}
	    \begin{figure}[htbp]
	        \centering
	        \subfloat[We]{
	            \includegraphics[width=0.45\linewidth]{../sample/graphics/anonymous.png}
	        } \qquad
	        \subfloat[Are]{
	            \includegraphics[width=0.45\linewidth]{../sample/graphics/anonymous.png}
	        }

	        \quad

	        \subfloat[Watching]{
	            \includegraphics[width=0.45\linewidth]{../sample/graphics/anonymous.png}
	        } \qquad
	        \subfloat[You]{
	            \includegraphics[width=0.45\linewidth]{../sample/graphics/anonymous.png}
	        }
	        \caption{We Are Watching}
	        \label{fig:waw}
	    \end{figure}
\end{document}